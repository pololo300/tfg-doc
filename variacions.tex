\section{Game variations}

\textsc{Tetris}'s inputs are the board and the sequence of pieces, but to define the problem all other parameters need to be specified. To define a game we need to provide: the possible board dimensions, polyominoes that will appear in the game, the available moves and the initial state function.

In order to present all variations we will first define the basic Tetris, the problem presented in \cite{TIH}. Then a Tetris variation is the basic Tetris problem, but with some changes on the parameters. Two variations can be combined to produce a new one. Variations are grouped by the parameters they change of the definition.

We are introducing also some notation to define different Tetris games. 

\subsection{Basic Tetris}

The problem is:

$$\textsc{Tetris}\left[ \mathbb{N} \times \mathbb{N},T_4,\varphi_0, \{s_l,s_r, r_+, r_-, d, f \},\Phi \right]$$

Let's break down the definition into parts. $\mathbb{N} \times \mathbb{N}$ indicates that the problem instances contain boards of arbitrary dimensions. $T_4$ indicates that all the instances use the classic set of pieces, the ones made up by joining 4 blocks. $\varphi_0$ is the initial state function that places the pieces in the top-middle of the board. The available moves in each match are: slide to the left, slide to de right, rotate clockwise, rotate counterclockwise, drop by one row and fix a piece. 

For the objective function $\Phi$, \clearing or \survival will be both considered.


\subsection{Pieces Variations}

These variations change the set of polyominoes $T$ of the problem. 

\begin{itemize}

  \item $k\textsc{-tris} := \textsc{Tetris}[T = T_k]$ 

  \item $\leq k\textsc{-tris} := \textsc{Tetris}[T = T_1 \cup \dots T_k]$

\end{itemize}

The first changes the classic $T_4$ polyominoes by $T_k$, the set that contains all the $k$-ominoes. The second one uses all the $q$-ominoes up to $k$. We can also consider variation with some specific pieces. For example, in \cite{TWFP}, the variation \emph{"Tetris with fewer pieces types"} uses subsets of $T_4$. We can write:

$$\textsc{Tetris}[\{ \TT, \OO \}] := \textsc{Tetris}[T = \{ \TT, \OO \}]$$

abusing the notation. The above problem is the basic Tetris restricting the tetrominoes to the two indicated pieces.


\subsection{Board Variations}

With boards, restricting the set of board dimensions, new variations appear.

\begin{itemize}

  \item $c\textsc{-column-Tetris} := \textsc{Tetris}[D = \mathbb{N} \times \{c\}]$
  \item $r\textsc{-row-Tetris} := \textsc{Tetris}[D = \{r\} \times \mathbb{N} ]$
\end{itemize}

In this case $c$\textsc{-column-Tetris} and $r$\textsc{-row-Tetris} instances are a subset of basic Tetris instances, only admitting instances where the board width, respectively the height, is $c$, respectively $r$.

\subsection{Moves Variations}

Move variations cover a wither range of different games, making it more difficult to classify. Inset we present popular variations. The most simple game variation is:

\begin{itemize}
  \item $\textsc{Tetris-NoRotation} := \textsc{Tetris}[M = \{s_l, s_r, d, f\}]$
\end{itemize}

As the name indicates, in the matches of this problem the player cannot rotate the pieces. The following variations are motivated by the video game.


When playing Tetris, specially in the initial rounds of a game, the player can hard drop a piece in order to avoid dropping the piece one row at a time. \emph{drop} and \emph{fix} are removed from the set of moves, and \emph{hard-drop} is added. Hard-dropping a piece moves it maximally downward before fixing into place. 

\begin{itemize}
  \item $\textsc{Tetris-HardDrop} := \textsc{Tetris}[M = \{ r_+, r_-, s_l, s_r, hd \}]$
\end{itemize}

In the video game the pieces fall faster as the player fixes pieces. This causes that, in large games, the player reaches a point where the pieces appear in the lowest empty postion (in the middle column).

The next variation simulates this situation. So instead of periodically moving down one unit, all pieces move maximally down-ward instantly, and the player is not allowed to control how fast a piece moves downward. The player is still free to rotate or move the piece left or right before the piece locks. \emph{drop} move is removed, and the vertical position of the initial state of a piece is now the lowest possible.

\begin{itemize}
  \item $\textsc{Tetris-20G} := \textsc{Tetris}[\varphi = \varphi_{20g}, M = \{ r_+, r_-, s_l, s_r, f \}]$
\end{itemize}

Where $\varphi_{20g}$ is the mentioned initial state function. 

\vspace{10px}
Finally, in some video game implementations, the player has a box to put aside a piece for a later use. To define this variation we need to add this function to the set of available moves.

\textbf{NOTA:} no es una funcio propiament, ja que pot retornar diferents valors per una mateixa entrada. No se com fer-ho encaixar.

