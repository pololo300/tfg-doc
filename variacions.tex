\section{Game variations}
% TODO:

\textsc{Tetris}'s inputs are the board and the sequence of pieces, but to define the problem all other parameters need to be specified. To define a game we need to provide: the polyominos that will appear in the game, the available moves and the initial state function.

In order to present all variations we will first define the basic Tetris, the problem presented in \cite{TIH}. Then a Tetris variation is the basic problem, but with some changes on the parameters. Two variations can be combined to produce a new one. Variations are grouped by the parameters they tweak.

We are introducing also some notation to define different Tetris games. 

\subsection{Basic Tetris}

The problem is:

$$\textsc{Tetris}\left[T_4,\varphi_0, \{s_l,s_r, r_+, r_-, d, f \},\Phi \right]$$

Let's break down the definition into parts. $T_4$ indicates that all the instances use the classic set of pieces, the ones made up by joining 4 blocks. $\varphi_0$ is the initial state function that places the pieces in the top-middle of the board. The available moves in each match are: slide to the left, slide to de right, rotate clockwise, rotate counterclockwise, drop by one row and fix a piece. 

For the objective function $\Phi$, \clearing\ or \survival\ will be both considered.


\subsection{Pieces Variations}

These variations change the set of polyominoes $T$ of the problem. 

\begin{itemize}

  \item $k\textsc{-tris} := \textsc{Tetris}[T = T_k]$ 

  \item $\leq k\textsc{-tris} := \textsc{Tetris}[T = T_1 \cup \dots T_k]$

\end{itemize}

The first changes the classic $T_4$ polyominoes by $T_k$, the set that contains all the $k$-ominoes. The second one uses all the $q$-ominoes up to $k$. We can also consider variation with some specific pieces. For example, in \cite{TWFP}, the variation uses subsets of $T_4$. We can write:

$$\textsc{Tetris}[\{ \TT, \OO \}] := \textsc{Tetris}[T = \{ \TT, \OO \}]$$

abusing the notation. The above problem is the basic Tetris restricting the tetrominos to the two indicated pieces.


\subsection{Board Variations}

With boards, restricting the set of board dimensions, new variations appear.

\begin{itemize}

  \item $c\textsc{-column-Tetris} := \textsc{Tetris}$ where all input boards have $c$ columns.
  \item $r\textsc{-row-Tetris} := \textsc{Tetris}$ where all input boards have $r$ rows.
\end{itemize}

In this case $c$\textsc{-column-Tetris} and $r$\textsc{-row-Tetris} instances are a subset of basic Tetris instances, only admitting instances where the board width, respectively the height, is $c$, respectively $r$.

\subsection{Moves Variations}

Move variations cover a wither range of different games, making it more difficult to classify. Usually consists on adding or removing moves to the standard set. Inset we present popular variations. The most simple game variation is:

\begin{itemize}
  \item $\textsc{Tetris-NoRotation} := \textsc{Tetris}[M = \{s_l, s_r, d, f\}]$
\end{itemize}

As the name indicates, in the matches of this problem the player cannot rotate the pieces. When playing Tetris, specially in the initial rounds of a game, the player can hard drop a piece in order to avoid dropping the piece one row at a time. Hard-dropping a piece moves it maximally downward before fixing into place. \emph{drop} and \emph{fix} are removed from the set of moves, and \emph{hard-drop} is added.  

\begin{itemize}
  \item $\textsc{Tetris-HardDrop} := \textsc{Tetris}[M = \{ r_+, r_-, s_l, s_r, hd \}]$
\end{itemize}

In some video game implementations, players have the ability to set aside a piece for later use \cite{WikiFandom}. By adding a \emph{hold function} to the set of available moves, we can define a variation known as \textsc{Tetris-WithHold}. However, this functionality will not be explored at this time.

\vspace{1em}


In the video game the pieces fall faster as the player fixes pieces. This causes that, in long matches, the player reaches a point where the pieces are pushed maximally downwards after each move. This behavior can be achieved by composing each function of the moves set with soft drop, so each time a move is done we ensure the piece is placed maximally downwards.   
\begin{itemize}
  \item $\textsc{Tetris-20G} := \textsc{Tetris}[\varphi = d_s \circ \varphi_0, M = d_s \circ M]$
\end{itemize}

The initial state function also is modified to ensure that the piece also is given to the player in the lowest possible possition. 
