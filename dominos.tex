\chapter{Dominoes}

The objective is to determine the sources of complexity in Tetris. It has been shown\cite{TT, TCB} that increasing the number of cells that form the pieces adds complexity to the game. Consequently, analyzing simpler versions of the game can provide valuable insights. To achieve this, we will focus on studying Tetris with dominoes. This variation is particularly interesting because, when rotation is allowed, there is only one polyomino in the game: the domino. However, when rotation is not allowed, there are effectively two distinct but simple pieces.

As shown in Table~\ref{tab:tt}, the variation \textsc{2-{Tris-NoRotation}} \clearing\ is \npc\cite{TT}. The other 3 variations remain an open problem. This section begins defining the game components, reviews the proof of the \nph\ result and finally explores some of the variations.

\section{Definitions}

Before proceeding further, the missing gaps of the general game definition need to be filled, in order to fit the problem into the previous definition. Starting with how domino piece states are mapped onto the board and how the pieces move. A domino state \( \piece[\VD][\theta][i][j][f] \) is mapped into:

\begin{center}
\begin{equation}
\piece[\VD][\theta][i][j][f] \mapsto  \begin{cases}
    \{ \cell, \cell[i][j+1] \} &\text{, } \theta = 0^\circ\\
    \{ \cell, \cell[i-1][j] \} &\text{, } \theta = 90^\circ\\
    \{ \cell, \cell[i][j-1] \} &\text{, } \theta = 190^\circ\\
    \{ \cell, \cell[i+1][j] \} &\text{, } \theta = 270^\circ
\end{cases}
\end{equation}
\end{center}

The mapping works like a clock: the piece's position serves as the center, and the second cell acts as the clock's handle. Figure~\ref{dom:mapping} illustrates the mapping for all orientations.

\begin{figure}[h]
    \centering
    \includegraphics[width=0.6\textwidth]{./pictures/dominoes/mapping.pdf}
    \caption{The mapping of a piece placed at the center of the grid. From left to right, the orientations are \(0^\circ\), \(90^\circ\), \(180^\circ\), and \(270^\circ\), respectively.}
    \label{dom:mapping} 
\end{figure}

The moves for the dominoes contains the default drop $d$ and slides $s_r, s_l$ . For the fix move the \emph{partial lock out rule}\cite{WikiFandom} will be assumed, fixing only a piece if is entirely  inside the board. Without the rule, variants like \survival\ can be trivially solved when rotation is allowed.

Regarding rotation moves, most modern Tetris implementations employ the Super Rotation System (SRS), the official Tetris Guideline standard for tetromino rotation behavior\cite{SRS}. In SRS, when a piece is rotated and overlaps with filled cells, the system attempts to reposition the piece by testing a predefined set of translations. 

\vspace{1em}

Following a similar approach, Dominoes Rotation System (DRS) is defined, described in Table~\ref{dom:rotation}. Intuitively, DRS tries to rotate a piece by moving it slightly depending on its orientation: \(r_+\) attempts to place the piece to the right or above, while \(r_-\) shifts it to the left or below, depending on its orientation. When a piece is rotated, DRS performs the following steps:
\begin{enumerate}
    \item First, DRS attempts the rotation using Test 1. 
    \item If the resulting piece overlaps with a filled cell, DRS proceeds to Test 2. 
    \item If Test 2 also fails, the piece cannot be rotated, and the move is deemed illegal.
\end{enumerate}


\begin{table}[ht]
\centering
\begin{tabular}{|c || c | c || c | c ||} 
 \hline
  & \multicolumn{2}{| c ||}{ $\VD$ } & \multicolumn{2}{| c ||}{$\HD$} \\
 \hline               
 & Test 1  & Test 2 & Test 1  & Test 2 \\ 
 \hline               
 $r_+$ & $\vcenter{\hbox{\includegraphics[scale=0.3]{./pictures/dominoes/rotation/vert_clock_1.pdf}}}$ & $\vcenter{\hbox{\includegraphics[scale=0.3]{./pictures/dominoes/rotation/vert_clock_2.pdf}}}$  & $\vcenter{\hbox{\includegraphics[scale=0.3]{./pictures/dominoes/rotation/horit_clock_1.pdf}}}$  & $\vcenter{\hbox{\includegraphics[scale=0.3]{./pictures/dominoes/rotation/horit_clock_2.pdf}}}$ \\ 
 \hline                             
 $r_-$ & $\vcenter{\hbox{\includegraphics[scale=0.3]{./pictures/dominoes/rotation/vert_anti_1.pdf}}}$ & $\vcenter{\hbox{\includegraphics[scale=0.3]{./pictures/dominoes/rotation/vert_anti_2.pdf}}}$  & $\vcenter{\hbox{\includegraphics[scale=0.3]{./pictures/dominoes/rotation/horit_anti_1.pdf}}}$  & $\vcenter{\hbox{\includegraphics[scale=0.3]{./pictures/dominoes/rotation/horit_anti_2.pdf}}}$ \\ 
 \hline
\end{tabular}
\caption{The DRS rotation table. The top rows indicate the initial orientation of the piece: vertical (\(\VD\)) or horizontal (\(\HD\)). The left columns specify the rotation direction (\(r_+\) for clockwise, \(r_-\) for counterclockwise), and the tests describe possible placements. In each image, the original piece is shown in white, while the resulting piece is displayed over it.}
\label{dom:rotation}
\end{table}


Figure~\ref{dom:drs} provides an example of the DRS rotation system in action.

\begin{figure}[ht]
  \centering
  \begin{subfigure}[b]{0.15\textwidth}
    \centering
    \includegraphics[width=0.9\textwidth]{pictures/dominoes/drs-1.pdf}
    \caption{}
  \end{subfigure}
  \begin{subfigure}[b]{0.15\textwidth}
    \centering
    \includegraphics[width=0.9\textwidth]{pictures/dominoes/drs-2.pdf}
    \caption{}
  \end{subfigure}
  \begin{subfigure}[b]{0.15\textwidth}
    \centering
    \includegraphics[width=0.9\textwidth]{pictures/dominoes/drs-3.pdf}
    \caption{}
  \end{subfigure}
  \caption{When rotating clockwise the domino in (a) DRS first ties to place it like (b), but this position isn't legal. Test 2 tries to place it in the right column which results a valid position.}
  \label{dom:drs}
\end{figure}

DRS allows pieces to move freely across the board, enabling not only standard movements but also upward motion in a staircase-like pattern, as shown in Figure~\ref{dom:staircase}. Most importantly, DRS makes it possible to place a domino in almost any position, despite the board configuration.

\begin{figure}
  \centering
  \includegraphics[width=0.3\textwidth]{pictures/dominoes/staricase.pdf}
  \caption{The overvevirew of a tragjectory starting at the left to the rightmost piece. The moves sequece is: $(3 \times s_r, 8 \times r_+, 2 \times s_r)$}
  \label{dom:staircase}
\end{figure}

\begin{definition}
  Let $B$ be a board in any configuration. A \emph{path} $p = (c_1,\dots , c_k) $ is a sequence of $k$ adjacent unfilled cells of $B$ where $c_1$ is a cell from the top row.
\end{definition}

\begin{lemma} 
  Given a board $B$, exists a trajectory $\sigma$ for a domino that follows any path $p = (c_1,\dots , c_k) $ satisfying for any $ 1 < l < k $:  \label{lemma:paths}
\begin{enumerate}
  \item if \(c_l = \cell\), \( c_{l+1} = \cell[i+1][j] \) and \( c_{l+2} = \cell[i+2][j] \), then either 
    \begin{enumerate}
      \item $\cell[i][j-1]$ is filled and cell $\cell[i+1][j-1]$ unfilled or \label{dom:path:up-left}
      \item $\cell[i][j+1]$ is filled and cell $\cell[i+1][j+1]$ unfilled \label{dom:path:up-right}
    \end{enumerate}
  \item $  c_l = \cell$, \( c_{l+1} = \cell[i+1][j] \) and \( c_{l+2} = \cell[i+1][j \pm 1] \), cell $\cell[i][j \pm 1]$ of $B$ is filled, respectively. \label{dom:path:turn} 
\end{enumerate}
\end{lemma}
\begin{proof}
  Let \( p = (c_1, c_2, \dots, c_k) \) be such a path. Assuming that the domino is initially placed in cells \( c_1 \) and \( c_2 \) since the path starts at the top of the board. By proving that given a domino placed in \( c_l \) and \( c_{l+1} \) there exists a move, or a sequence of moves, that places the domino in \( c_{l+1} \) and \( c_{l+2} \) the entire trajectory can be constructed. 

  If $c_l = c_{l+2}$ no moves are needed. If \( c_l \), \( c_{l+1} \), and \( c_{l+2} \) are in the same row, a slide to the right $s_r$ or to the left $s_l$ places the domino in \( (c_{l+1}, c_{l+2}) \). When cells are in the same column and path goes downwards a drop $d$ does the job. If the path goes upwards, \ref{dom:path:up-left} or \ref{dom:path:up-right} must hold. Assuming \ref{dom:path:up-left} holds, the domino is moved from $c_l, c_{l+1}$ to $c_{l+1}, c_{l+2}$ with the following moves sequence:
  \[
    \vcenter{\hbox{\includegraphics[width=0.05\textwidth]{pictures/dominoes/turns/up-1-0.pdf}}}
    \xrightarrow{r_-}
    \vcenter{\hbox{\includegraphics[width=0.05\textwidth]{pictures/dominoes/turns/up-1-1.pdf}}}
    \xrightarrow{r_+}
    \vcenter{\hbox{\includegraphics[width=0.05\textwidth]{pictures/dominoes/turns/up-1-2.pdf}}}
    \xrightarrow{s_r}
    \vcenter{\hbox{\includegraphics[width=0.05\textwidth]{pictures/dominoes/turns/up-1-3.pdf}}}
    \hspace{3em}
    \vcenter{\hbox{\includegraphics[width=0.05\textwidth]{pictures/dominoes/turns/up-2-0.pdf}}}
    \xrightarrow{r_-}
    \vcenter{\hbox{\includegraphics[width=0.05\textwidth]{pictures/dominoes/turns/up-2-1.pdf}}}
    \xrightarrow{r_+}
    \vcenter{\hbox{\includegraphics[width=0.05\textwidth]{pictures/dominoes/turns/up-2-2.pdf}}}
  \]

  The sequence of moves depends on the neighboring cells. When condition \ref{dom:path:up-right} is satisfied, the sequence $(r_+, r_+)$ places the domino in $c_{l+1}$ and $c_{l+2}$, regardless of the neighboring cells.

  Finally, the remaining cases are those where the path turns. There are four possible turns and each can be traversed in both directions, so there are eight scenarios. as shown in Figure~\ref{dom:turns}.

\begin{figure}[h]
  \centering
  \begin{subfigure}[b]{0.1\textwidth}
    \centering
    \includegraphics[width=0.9\textwidth]{pictures/dominoes/turns/turn_1.pdf}
    \caption{}
    \label{dom:turn1}
  \end{subfigure}
  \begin{subfigure}[b]{0.1\textwidth}
    \centering
    \includegraphics[width=0.9\textwidth]{pictures/dominoes/turns/turn_2.pdf}
    \caption{}
    \label{dom:turn2}
  \end{subfigure}
  \begin{subfigure}[b]{0.1\textwidth}
    \centering
    \includegraphics[width=0.9\textwidth]{pictures/dominoes/turns/turn_3.pdf}
    \caption{}
    \label{dom:turn3}
  \end{subfigure}
  \begin{subfigure}[b]{0.1\textwidth}
    \centering
    \includegraphics[width=0.9\textwidth]{pictures/dominoes/turns/turn_4.pdf}
    \caption{}
    \label{dom:turn4}
  \end{subfigure}
  \begin{subfigure}[b]{0.1\textwidth}
    \centering
    \includegraphics[width=0.9\textwidth]{pictures/dominoes/turns/turn_5.pdf}
    \caption{}
    \label{dom:turn5}
  \end{subfigure}
  \begin{subfigure}[b]{0.1\textwidth}
    \centering
    \includegraphics[width=0.9\textwidth]{pictures/dominoes/turns/turn_6.pdf}
    \caption{}
    \label{dom:turn6}
  \end{subfigure}
  \begin{subfigure}[b]{0.1\textwidth}
    \centering
    \includegraphics[width=0.9\textwidth]{pictures/dominoes/turns/turn_7.pdf}
    \caption{}
    \label{dom:turn7}
  \end{subfigure}
  \begin{subfigure}[b]{0.1\textwidth}
    \centering
    \includegraphics[width=0.9\textwidth]{pictures/dominoes/turns/turn_8.pdf}
    \caption{\label{dom:turn8}}
  \end{subfigure}
    \caption{The possible turns. The domino before the move is painted, and the destination cell is marked in white. The dark gray cell represents the corner of the turn, which alters the sequence of moves depending on whether it is filled or not.} 
    \label{dom:turns} 
\end{figure}
The trajectory required to perform the move involves either one or two steps, depending on the state of the corner cell. The remaining cells can be in any configuration. Table~\ref{dom:turns-table} shows the necessary moves for each scenario based on whether the corner cell is filled.

\begin{table}[ht]
\centering
\begin{tabular}{|c || c | c | c | c | c | c | c | c |} 
 \hline
  & \ref{dom:turn1} & \ref{dom:turn2} & \ref{dom:turn3} & \ref{dom:turn4} & \ref{dom:turn5} & \ref{dom:turn6} & \ref{dom:turn7} & \ref{dom:turn8} \\
 \hline               
  filled & $ r_-  $ & $      r_-    $ & $   r_+       $ & $   r_+       $ & $   r_+       $ & $     r_-     $ & $   r_-       $ & $   r_+      $  \\
 \hline               
unfilled & $ r_-  $ & $     -       $ & $   r_+, s_r  $ & $   r_+       $ & $    -        $ & $  r_-, s_r   $ & $   r_-       $ & $   r_+      $  \\
 \hline               

\end{tabular}
\caption{The moves needed to do each turn of Table~\ref{dom:turns}. When the corner cell is unfilled in \ref{dom:turn2} and in \ref{dom:turn5}, the condition \ref{dom:path:turn} is not satisfied. }
\label{dom:turns-table}
\end{table}

Each step can be calculated in constant time, so the path can be computed in $\mathcal{O}(k)$.

\end{proof}


\section{Constructible Boards}

In this work is assumed that boards initial configurations can be built within the game rules, starting from an empty board. The set of constructible configurations for a game variation is bigger as the game has more elements. For $1$\textsc{-Tris} all the boards look like a histogram, since each piece occupies one cell, and for the standard Tetris game almost all configuration are constructible \cite{HCTC}. Next follows a result for constructible boards in dominoes games that will be used in later proofs.

\begin{lemma} \label{lemma:floating}
  In any constructible board, for any filled cell $\cell$ at least on of the downward neighbors ($\cell[i-1][j-1]$, $\cell[i-1][j]$ and $\cell[i-1][j+1]$) must be filled.
\end{lemma}  

\begin{proof}  
Let \(B\) be an \(n \times m\) board containing such a filled cell. Consider the sequence of boards \(B_0, B_1, \dots, B_k = B\), where \(B_0\) is an empty board and \(B_k = B\). Assume, without loss of generality, that \(B_k\) is the first board in the sequence containing a filled cell \(\cell_k\) (the cell \(\cell\) in \(B_k\)) such that the cells \(\cell[i-1][j-1]_k\), \(\cell[i-1][j]_k\), and \(\cell[i-1][j+1]_k\) are all unfilled.


The cell \(\cell_k\) cannot result from adding a domino that does not clear a row. Therefore, \(B_k\) must be obtained by adding a domino to \(B_{k-1}\) in such a way that it clears a row. Considering the row, the last placed domino, and the pieces used to achieve this, there are five possible ways of clearing a row:

\begin{figure}[ht]
  \centering
  \begin{subfigure}[b]{0.15\textwidth}
    \centering
    \includegraphics[width=0.9\textwidth]{./pictures/dominoes/proff-floating/clear-row-1.pdf}
    \caption{\label{dom:proff-floating:clear1}}
  \end{subfigure}
  \begin{subfigure}[b]{0.15\textwidth}
    \centering
    \includegraphics[width=0.9\textwidth]{./pictures/dominoes/proff-floating/clear-row-2.pdf}
    \caption{\label{dom:proff-floating:clear2}}
  \end{subfigure}
  \begin{subfigure}[b]{0.15\textwidth}
    \centering
    \includegraphics[width=0.9\textwidth]{./pictures/dominoes/proff-floating/clear-row-3.pdf}
    \caption{\label{dom:proff-floating:clear3}}
  \end{subfigure}
  \begin{subfigure}[b]{0.15\textwidth}
    \centering
    \includegraphics[width=0.9\textwidth]{./pictures/dominoes/proff-floating/clear-row-5.pdf}
    \caption{\label{dom:proff-floating:clear5}}
  \end{subfigure}
  \begin{subfigure}[b]{0.15\textwidth}
    \centering
    \includegraphics[width=0.9\textwidth]{./pictures/dominoes/proff-floating/clear-row-4.pdf}
    \caption{\label{dom:proff-floating:clear4}}
  \end{subfigure}
\end{figure}


  Let $r$ be the cleared row in \( B_{k-1} \), bottom cleared row in case \ref{dom:proff-floating:clear4}. Consider the board $B_{k-1}$ after placing the domino but before clearing the lines. Each cell $\cell[i'][j']$ from this board has the same downward neighbors as the cell $\cell[i'+1][j']_k$ ($i+2$ in case \ref{dom:proff-floating:clear4}) except for the cells just above the cleared row(s). Hence, cell \(\cell_k \) must be in row $r$: \( \cell_k = \cell[r]_k \).

  Any filled cell in $B_{k-1}$ has one of its downward neighbors filled, so if $\cell[r]_{k-1}$ is filled then $\cell[r]_k$ would have a downward neighbor filled, because cells below row $r$ aren't modified when clearing the row. Therefore, $\cell[r]_k$ can't be in a cell filled in $B_k$. This leads to few options, show in Figure~\ref{dom:proff-floating:final-clear}.

  \begin{figure}[ht] \label{dom:proff-floating:final-clear}
  \centering
  \begin{subfigure}[b]{0.15\textwidth}
    \centering
    \includegraphics[width=0.9\textwidth]{./pictures/dominoes/proff-floating/scenario-1.pdf}
    \caption{}
    \label{floating:a}
  \end{subfigure}
  \begin{subfigure}[b]{0.15\textwidth}
    \centering
    \includegraphics[width=0.9\textwidth]{./pictures/dominoes/proff-floating/scenario-2.pdf}
    \caption{}
    \label{floating:b}
  \end{subfigure}
  \begin{subfigure}[b]{0.15\textwidth}
    \centering
    \includegraphics[width=0.9\textwidth]{./pictures/dominoes/proff-floating/scenario-3.pdf}
    \caption{}
    \label{floating:c}
  \end{subfigure}
  \begin{subfigure}[b]{0.15\textwidth}
    \centering
    \includegraphics[width=0.9\textwidth]{./pictures/dominoes/proff-floating/scenario-4.pdf}
    \caption{}
    \label{floating:d}
  \end{subfigure}
  \begin{subfigure}[b]{0.15\textwidth}
    \centering
    \includegraphics[width=0.9\textwidth]{./pictures/dominoes/proff-floating/scenario-5.pdf}
    \caption{}
    \label{floating:e}
  \end{subfigure}
  \linebreak
  \begin{subfigure}[b]{0.15\textwidth}
    \centering
    \includegraphics[width=0.9\textwidth]{./pictures/dominoes/proff-floating/scenario-6.pdf}
    \caption{}
    \label{floating:f}
  \end{subfigure}
  \begin{subfigure}[b]{0.15\textwidth}
    \centering
    \includegraphics[width=0.9\textwidth]{./pictures/dominoes/proff-floating/scenario-7.pdf}
    \caption{}
    \label{floating:g}
  \end{subfigure}
  \caption{Each picture represents the hole in row $r$ after placing the domino and before clearing the row. In yellow the cell candidate to fill $\cell[r]_k$. Blue cells correspond to the domino placed.}
\end{figure}

After clearing the row, none of the above cases succeed in creating the cell without downward neighbors, so such a cell can't appear in any board.
\end{proof}

Moreover, this lemma implies that in any constructible configuration, completely floating blocks--i.e., blocks whose lower corners do not touch any domino--cannot exist. 

\begin{corollary} \label{coro:column}
  In any constructible configuration, each filled cell forms an uncrossable column down to the bottom of the board.
\end{corollary}
\begin{proof}
  If a cell is filled, by Lemma~\ref{lemma:floating}, one of its downward neighbors is filled. And his downward neighbor filled also has a downward neighbor filled, and so on until the bottom of the board. Due to the shape of the column, it cannot be crossed by any path.
\end{proof}

This characterization also provides insight into the possible arrangements of holes and paths within the board, making it useful for following proofs. The following corollary will help when filling the board. 

\begin{corollary} \label{coro:no-up-path}
  In any constructible configuration, for any path $p = (c_1, \dots, c_k)$ exists another $p' = (d_1, \dots, d_{k'})$ that starts and ends on same cells while not going upwards. Formally:
  \begin{enumerate}
    \item $c_1 = d_1$, $c_2 = d_2$, $c_{k-1} = d_{k'-1}$ and $c_k = d_{k'}$.
    \item for any $ 1 < l < k'-1$, if $d_l = \cell$ then $d_{l+1} \neq  \cell[i+1]$. \label{lemma:paths:cond2}
  \end{enumerate}

\end{corollary}
\begin{proof}
  Let $p = (c_1, \dots, c_k)$ be such a path, and $c_l$ the last cell such that $c_l = \cell$ and $c_{l+1} = \cell[i+1]$. Let $c_m$ be the last cell of $p$ before $c_{l+1}$ at row $i+1$. Consider all cells in row $i+1$ between $c_l$ and $c_m$, as is show in the picture:

  $$ \includegraphics[width=0.4\textwidth]{./pictures/dominoes/equivalent-path.pdf} $$

  If any of these cells are filled, by Corollary~\ref{coro:column}, no path could go under row $i+1$, hence between cells are empty. If $c_l = \cell[i+1][j']$, the new path that goes straight thought these cells is: 

  $$ p' = (c_1, \dots, c_m = \cell[i+1][j'], \cell[i+1][j'+1], \dots, \cell[i+1][j-1], \cell[i+1]= c_l, \dots c_k)$$

  when $ j' < j $, and analogously for $j < j'$. Repeating the argument for $p'$ until condition~\ref{lemma:paths:cond2} is satisfied the equivalent path is obtained.
\end{proof}


With the previous results a domino can be placed to any position of the board, but not fixed since the below cells can be unfilled. 

\section{Tetris Survival with Rotation}

The problem $\textsc{Tetris}\lbrack \VD \rbrack $ \survival\ can be formulated as follows: \emph{Given an arbitrarily sized board with an constructible initial configuration and a sequence of \( k \) dominoes, is there a way to play all the pieces while avoiding losing?} As pointed out\cite{TT}, if there exists a strategy to clear a single row, it is possible to survive indefinitely by employing the following piece-placement strategy:

\begin{enumerate}
    \item Rotate the piece to be vertical. 
    \item Place the piece in any column with the two top cells empty.
\end{enumerate}

To solve the problem it must be demonstrated that deciding whether any row can be cleared and, when not, compute the maximum number of pieces that can be fixed into the board before losing can both be computed in polynomial time. The problem will be addressed in these three parts separately, and then the results will be combined.  

\vspace{1em}
To clear a row, dominoes must be fixed into specific board positions. The following result shows that determining whether a domino can be placed in a given position can be checked in polynomial time.

\begin{lemma} \label{lemma:can-fix}
 Let $B$ be an $n\times m $ board, and let $p$ be a path in $B$. Checking if exists a match $\Sigma = B,\sigma_1, B_1, \sigma_2, \dots, \sigma_k, B_k$ such that $\sigma_k$ fixes a domino in any position of the path $p$ can be computed in $\mathcal{O}(n)$. If the position is horizontally orientated the answer is always yes.
\end{lemma}
\begin{proof} 
  Let $(c_1, c_2)$ be the desired position in the path. Without loss of generality assume that the path $p$ ends with $(c_1, c_2)$.  Also assume that $p$ doesn't go upwards, by considering the equivalent path provided by Lemma~\ref{coro:no-up-path}.

  Since $(c_1, c_2)$ are in the end of a path, by Lemma~\ref{lemma:paths} exists a trajectory $\sigma$ that places a domino in $(c_1, c_2)$. The piece now needs to be fixed. If the cell below $c_1$ or the cell below $c_2$ is filled the domino can be fixed directly. Otherwise, when the domino can't be fixed one of the below pieces has to be filled. When the domino is horizontally orientated it can be fixed by repeteadly placing the domino in $(c_1, c_2)$ with $\sigma$ and hard dropping it eventually a domino will be fixed in $(c_1, c_2)$.

  \[
    \vcenter{\hbox{\includegraphics[width=0.04\textwidth]{pictures/dominoes/proff-fix/horti-1.pdf}}}
    \xrightarrow{d_h}
    \vcenter{\hbox{\includegraphics[width=0.04\textwidth]{pictures/dominoes/proff-fix/horti-2.pdf}}}
    \xrightarrow{d_h}
    \vcenter{\hbox{\includegraphics[width=0.04\textwidth]{pictures/dominoes/proff-fix/horti-3.pdf}}}
    \xrightarrow{d_h}
    \vcenter{\hbox{\includegraphics[width=0.04\textwidth]{pictures/dominoes/proff-fix/horti-4.pdf}}}
    \xrightarrow{d_h}
    \vcenter{\hbox{\includegraphics[width=0.04\textwidth]{pictures/dominoes/proff-fix/horti-5.pdf}}}
  \]

  If $(c_1, c_2)$ are in row $i$ and the first filled cell bellow $c_1$ and $c_2$ is in row $i'$, the match is:

  $$ \Sigma = B, (\sigma, d_h), B_1, \dots,B_{i-i'-1},  (\sigma, d_h), B_{i-i'}  $$

  Where the hard drop can be decomposed into the sequence of drops and fix if isn't available in the move set. In this scenario the domino can be always fixed in $(c_1, c_2)$.

  Finally, the case where $(c_1, c_2)$ are vertically orientated. If $c_1 = \cell$ and $c_2 = \cell[i+1]$, the cell $\cell[i-1]$ must be also filled to fix the domino in $(c_1, c_2)$. This can be achived by placing another domino into one of the following:
\[
  \begin{array}{c@{\hspace{-0.5em}}cc@{\hspace{-0.5em}}cc@{\hspace{-0.5em}}c}
    \begin{minipage}{0.08\textwidth}
      \includegraphics[width=\textwidth]{pictures/dominoes/proff-fix/vert-1.pdf}
    \end{minipage} &
    \begin{minipage}{0.2\textwidth}
      \center
      \(\cell[i-1][j-1], \cell[i-1][j]\)
    \end{minipage} &
    \begin{minipage}{0.08\textwidth}
      \includegraphics[width=\textwidth]{pictures/dominoes/proff-fix/vert-3.pdf}
    \end{minipage} &
    \begin{minipage}{0.15\textwidth}
      \center
      \(\cell[i-1][j], \cell[i-2][j]\)
    \end{minipage} &
    \begin{minipage}{0.08\textwidth}
      \includegraphics[width=\textwidth]{pictures/dominoes/proff-fix/vert-2.pdf}
    \end{minipage} &
    \begin{minipage}{0.2\textwidth}
      \center
      \(\cell[i-1][j], \cell[i-1][j+1]\)
    \end{minipage}
  \end{array}
\]
 
  If any of the before dominoes can be fixed into its position, then domino can be fixed into $c_1,c_2$. The overall procedure is the following: 


  \begin{algorithm}
  \begin{algorithmic}
    \Procedure{CanBeFixed}{$B, c_1 = \cell, c_2 = \cell[i'][j']$}\Comment{$c_1, c_2$ adjacent cells in $B$}
      \If{$\cell[i][j]$ or $\cell[i'][j']$ filled} 
        \State \Return False  \Comment{$c_1$ or $c_2$ are filled}
      \ElsIf{$\cell[i-1][j]$ or $\cell[i'-1][j']$ unfilled}
        \State \Return True \Comment{the domino can be directly fixed}
      \ElsIf{$ i = i'$}
        \State \Return True \Comment{an horizontal domino can always be placed below}
      \EndIf
      \State $c_2 = \cell[i+1][j]$ \Comment{$c_1$ and $c_2$ are vertically aligned}
      \State \Return \Call{CanBeFixed}{$B$, $\cell[i-1]$, $\cell[i-2]$} 
    \EndProcedure
  \end{algorithmic}
  \end{algorithm}


  In worst case scenario the algorithm does $n / 2$ calls of constant cost, since in each call the row is decreased by two. Hence, the total cost is $\mathcal{O}(n)$.

\end{proof}

Note that the proof doesn't ensure that no row is cleared when filling the board, this is because the goal is to fully fill a row. If at some point a row is cleared then the goal would be acomplished. Now we are ready to solve the main question of $\textsc{Tetris}\lbrack \VD \rbrack $ \survival.

\begin{lemma} \label{dom:clear-top}
For any $n \times m$ board, deciding whether the top row can be cleared is in \pp.
\end{lemma}
\begin{proof}
Consider the top row of an \( n \times m \) board in a given configuration, and split the row into groups of adjacent empty cells, which will henceforth be referred to as \emph{holes}. To clear the top row, it must be fully filled, and this requires that all its holes are fully filled. 

A hole of even length can always be filled entirely by placing horizontal dominoes, as guaranteed by Lemma~\ref{lemma:can-fix}. 

When the length of a hole is odd, one vertical domino must be placed. Enumerating the cells of the hole starting from 1, the vertical domino must be placed in any odd-positioned cell to avoid requiring additional vertical dominoes. Verifying whether the vertical domino can be placed in a positotion has cost of \( \mathcal{O}(n) \), again by Lemma~\ref{lemma:can-fix}. 

Since there are at most \( \mathcal{O}(m) \) holes in the top row, the total computational cost of filling all holes in the top row is \( \mathcal{O}(n \cdot m) \).

\end{proof}

\begin{lemma} \label{dom:clear-row}
For any $n \times m$ board, deciding if any row, except the first one, can be cleared is in \pp.
\end{lemma}

\begin{proof}
  To fill any row all its holes have to be filled, to do so there must exist a path that reaches any cell in the hole in order to fill it, if not the row can't be cleared. By Lemma~\ref{lemma:floating}, the hole and the row above can be in one of the following shapes:  

\begin{figure}[ht]
  \centering
  \begin{subfigure}[b]{0.24\textwidth}
    \centering
    \includegraphics[width=0.9\textwidth]{pictures/dominoes/row-holes/simple-hole-1.pdf}
    \caption{}
    \label{dom:fill-row:hole-a}
  \end{subfigure}
  \begin{subfigure}[b]{0.24\textwidth}
    \centering
    \includegraphics[width=0.9\textwidth]{pictures/dominoes/row-holes/simple-hole-2.pdf}
    \caption{}
    \label{dom:fill-row:hole-b}
  \end{subfigure}
  \begin{subfigure}[b]{0.24\textwidth}
    \centering
    \includegraphics[width=0.9\textwidth]{pictures/dominoes/row-holes/simple-hole-3.pdf}
    \caption{}
    \label{dom:fill-row:hole-c}
  \end{subfigure}
  \begin{subfigure}[b]{0.24\textwidth}
    \centering
    \includegraphics[width=0.9\textwidth]{pictures/dominoes/row-holes/simple-hole-4.pdf}
    \caption{}
    \label{dom:fill-row:hole-d}
  \end{subfigure}
  \label{dom:fill-row:holes}
  \caption{The possible shape of the holes in a row.}
\end{figure}

When the length of a hole is even, all holes can be filled by placing horizontal dominoes one by one, which, by Lemma~\ref{lemma:can-fix}, can always be done. When the length of a hole is odd, a vertical domino is required.

In hole~\ref{dom:fill-row:hole-a}, horizontal dominoes are placed from left to right, and the process is finished by placing a vertical domino at the right. Hole~\ref{dom:fill-row:hole-c} follows a similar approach, and hole~\ref{dom:fill-row:hole-d} can be filled in the same manner as either of the two previous cases. 

In a hole of type~\ref{dom:fill-row:hole-b} with length greater than 3 cells, dominoes can be placed from outside to inside, leaving a free cell in the middle to place the vertical domino as before. When the hole has length 3, it must be filled as follows:
$$ 
\includegraphics[width=0.11\textwidth]{./pictures/dominoes/row-holes/hole-3-1.pdf}
\hspace{3em}
\includegraphics[width=0.11\textwidth]{./pictures/dominoes/row-holes/hole-3-2.pdf}
$$

By Lemma~\ref{lemma:can-fix}, checking if any vertical dominoes can be placed has a cost of \( \mathcal{O}(n) \). The overall cost of filling a hole is thus \( \mathcal{O}(n) \), as the other dominoes can always be placed. 

In the worst-case scenario, all holes are of type~\ref{dom:fill-row:hole-b} with length 3, leading to a row cost of \( \mathcal{O}(m \cdot n) \). For any row except the first one, the overall cost is \( \mathcal{O}(m \cdot n^2) \), which is polynomial.

   
\end{proof}

In order to compute the maximum number of dominoes that can be placed into the board, the following result from Hopcroft and Karp\cite{MMBG} will be used:

\begin{theorem}[Hopcroft-Karp algorithm]\label{graph:Hopcroft-Karp}
  Let $G = (V,E)$ be a bipartite graph. There is anlgorithm that computes maximum matching in $\mathcal{O}(|E|\sqrt{|V|})$. 
\end{theorem}

\begin{lemma} \label{dom:max-fill}
  Let $B$ be a board such that no row can be cleared. The maximum number of dominoes that can be fixed into the board before loosing can be computed in $\mathcal{O}(??)$.
\end{lemma}

\begin{proof}
  Let $B$ be a board where no row can be cleared. Let $G = (V,E)$ the graph of reachable cells:
  \begin{equation*}
    \begin{split}
      V & = \{ \cell \mid \exists p \text{ path,  such that} \cell \in p \} \\
      E & = \{ (c_1, c_2) \mid c_1 \text{ adjacent } c_2 \} 
    \end{split}
  \end{equation*}


    \begin{minipage}{0.4\textwidth}
        \centering
        \includegraphics[width=0.8\textwidth]{./pictures/dominoes/proof-max-fill/board.pdf}
    \end{minipage}
    \begin{minipage}{0.1\textwidth}
        \centering
        $\longrightarrow$
    \end{minipage}
    \begin{minipage}{0.4\textwidth}
        \centering
        \includegraphics[width=0.8\textwidth]{./pictures/graph.pdf}
    \end{minipage}

  \vspace{1em}
  
  When the board is empty, the graph \( G \) is a grid, and for any board configuration, \( G \) is a subgraph of the grid. Since a grid graph is bipartite, any subgraph of the grid is also bipartite. Therefore, the graph \( G \) corresponding to the board \( B \) is bipartite for any configuration of the board.

  By Theorem~\ref{graph:Hopcroft-Karp}, the maximum matching of \( G \) can be computed in \( \mathcal{O}(|E| \sqrt{|V|}) \). In this case, \( |V| = \mathcal{O}(n \cdot m) \) and \( |E| = \mathcal{O}(n \cdot m) \), so the maximum matching can be computed in \( \mathcal{O}((mn)^{3/2}) \).

  Let \( M \) denote the maximum matching of \( G \). Each edge in \( M \) corresponds to a pair of adjacent empty cells where a domino can be placed. It remains to show that all \( k \) dominoes can be placed on the board \( B \).

  Now, let us transform the matching \( M \) into a new matching, \( M' \), by removing all edges that correspond to pairs of cells above filled cells. \( M' \) retains the same number of dominoes, and each domino is placed above a filled cell, or another domino.

  By Lemma~\ref{coro:no-up-path}, a domino can be placed on each edge of \( M' \). To ensure that all dominoes can be placed in their respective positions, the order in which they are placed corresponds to their distance from the top row. Self-blocking is avoided because the domino placed furthest from the top can always be placed first without blocking any other dominoes. The distance of each domino can be computed by just exploring the board and the sorting can be done in polynomial time. So the overall cost is \( \mathcal{O}((mn)^{3/2}) \).

\end{proof}

\begin{theorem}
  \textsc{2-tris} \survival is in \pp.
\end{theorem}

\begin{proof}
  Let \( B \) be an \( n \times m \) board with a given constructible configuration, and let \( k \) be the number of dominoes to be placed. By Lemma~\ref{dom:clear-row} and Lemma~\ref{dom:clear-top}, it can be determined, in polynomial time, whether any row can be cleared by checking from bottom to top. If a row can be cleared, it is possible to survive indefinitely by placing the dominoes vertically in any column where the top cell is unfilled. Hence, the answer is "yes".

  If no row can be cleared, then by Lemma~\ref{dom:max-fill}, the maximum number of dominoes, \( k_{\text{max}} \), that can be fixed onto the board before losing can be computed. Therefore, if \( k \leq k_{\text{max}} \), the answer is "yes"; otherwise, it is "no."
\end{proof}

\section{Clearing without rotation}

The proof doesn't use exactly \textsc{3-Partition} for the reduction, it uses a similar problem. The reduction and takes a similar approach as the first Tetris proof\cite{TIH} and other later ones\cite{TT, TWFP, TCB, CTV}. The problem is defined as follows:
\vspace{1em}

\begin{problem}[\textsc{3-Partition with Distinct Integers}]
  
 Given a multiset of positive integers $A = \{ a_1, a_2, \dots, a_{3m} \}$ such that:

$$ \frac{t}{4} < a < \frac{t}{2}\;\; \forall a \in A \text{, where } t = \frac{1}{m} \sum_{i=1}^{3m} a_i $$ 

 Decide whether the set $A$ can be partitioned into $m$ subsets $D_1, \dots D_m$, each of size $3$ and summing $t$. 

$$ | D_i | = 3 \text{ and } \sum_{a \in D_i} a = t\text{   } 1 \leq i \leq m$$
\end{problem}

Hulett, Will and Woeginger\cite{3PART} proved that the problem is strongly \nph.

\begin{theorem} 
  \textsc{2-{Tris-NoRotation}} \clearing\ is \npc.
\end{theorem}

\begin{proof}
  \textbf{PROOF TO FINISH}

  Let \( A = \{ a_1, a_2, \dots, a_{3m} \} \) be an instance of the \textsc{3-Partition} problem with \( t \) as defined. Construct the corresponding instance of \textsc{2-Tris-NoRotation} as follows:

  The board \( B \) is made up attaching horizontally \( m \) \emph{buckets}. A bucket consists of five adjacent columns, each of height \( 3m + 2t \). The first and fifth column of a bucket are filled, the third one is empty and second and fourth columns have filled the following cells:
  $$ 
    \{ \cell[2][j] \mid j \leq 2t \text{ or } j \text{ odd}\} \cup
    \{ \cell[4][j] \mid j \leq 2t \text{ or } j \text{ even}\}
    $$
  the top $3m$ rows will be refered as blocking rows and the $2t$ bottom rows as filling rows. The number of empty cells is $3m + 2t + 3m = 2(3m + t)$, so the number of filled cells in the whole board is $2m(3m + t)$. In the reduction, each bucket corresponds to one of the answer sets $D_i$, which will be filled while playing the game. The board looks like: 
  \[
    \underbrace{
    \begin{array}{c@{\hspace{0px}}c@{\hspace{0px}}c@{\hspace{0px}}c}
      \underbrace{\vcenter{\hbox{\includegraphics[width=0.1\textwidth]{./pictures/dominoes/proff-nph/bucket.pdf}}}}_{\text{bucket}}
      &
      \underbrace{\vcenter{\hbox{\includegraphics[width=0.1\textwidth]{./pictures/dominoes/proff-nph/bucket.pdf}}}}_{\text{bucket}}
      &
      \cdots 
      &
      \underbrace{\vcenter{\hbox{\includegraphics[width=0.1\textwidth]{./pictures/dominoes/proff-nph/bucket.pdf}}}}_{\text{bucket}}  
    \end{array}
  }_{m \text{ buckets}}
  \]

 The piece sequence consists on a sequence of vertical ($\VD$) or horizontal ($\HD$) dominoes. Each $a_i$ is encoded in sequence of dominoes and the sequence is obtained by attaching each $a_i$ encoded. Each $a_i$ is encoded as:
 \[
  S(a_i) \mapsto S_i = 
  ( \underbrace{\HD, \dots, \HD}_{m-1}, \underbrace{\VD, \dots, \VD}_{a_i}, \HD )
 \]
 The first $m-1$ dominoes will be referred as the priming sequence, the next $a_i$ dominoes as filling sequence and finally the clearing piece. So the piece sequence for the game is $ S = S_1 S_2 \dots S_{3m}$. The total number of cells of the dominoes sequence is
 \[
  2 \sum_{a_i \in A} (m-1) + a_i + 1 = 6m^2 + 2\sum_{a_i \in A} a_i = 6m^2 + 2mt = 2m(3m+t)
 \]

 which corresponds to the number of empty cells of the board, hence in order to clear the hole board each domino must be placed in the board holes. Moreover, dominos of the filling sequence can only be placed at the bottom of the filling rows and the priming sequence in the top row of the blocking rows.

 The first $m-1$ pieces of the sequence are $\HD$ that must be placed in the top row. Not placing any of the dominos in such position forces to place the piece in one row above, making the clearing impossible. After placing the $m-1$ all the buckets except one have the top row filled. Let $j$ be this bucket.

 \begin{center}
 \includegraphics[width=0.7\textwidth]{./pictures/dominoes/proff-nph/top-row-filled.pdf}
 \end{center}

 Next follows $a_1$ vertical dominoes that can only be placed in the middle column of the $j$ bucket, because any other position forces to fill a cell in a row above. Finally, the clearing piece must be placed in the bucket $j$, by the same argument, clearing the top row. After placing the first $m + a_1$ pieces of the sequence, the ones corresponding to $S_1$, $a_i$ dominoes have been added to the bottom of a bucket $j$ and the first row cleared.

 $$ picture $$

 Next follows $S_2$ and so on. If at some point the filling sequence of some $S_i$ overfills the filling rows, a $\VD$ is placed in a blocking row impeding the afected blocking rows to be cleared by $\HD$, so clearing is impossible.

 \begin{center}
 \includegraphics[width=0.7\textwidth]{./pictures/dominoes/proff-nph/overfill.pdf}
 \end{center}



\end{proof}

\section{Towards Clearing}

The next objecive is \textsc{2-tris} \clearing, the problem won't be solved but some intermediate results will be shown. Starting with simpler game versions

\subsection{Tetris with Vertical Dominoes}

Using the notation introduced earlier, the problem can be described as \( \textsc{Tetris-NoRotation}[\VD] \) \clearing. The input consists of a sequence of vertical dominoes and an arbitrarily sized \( n \times m \) board in a constructible configuration. In this variation, the initial state function differs from the default orientation, as the pieces are required to be in a vertical orientation from the beginning.

First we will to characterize the constructible boards with $\VD$ pieces without rotation by exploring the configuration starting from an empty board. 

Vertical dominoes consist of two vertical adjacent cells, so for an empty board any trajectory fixes the piece in the bottom row, filling $\cell[1][i]$ and $\cell[2][i]$ cells for any $1 \leq i \leq m$. The next domino can either go to an empty column or to the one before. Placing the first $m$ dominoes in unfilled columns clears the two lowest rows, and consequently the board. When a domino is placed in a non-empty column $i$, the $\cell[3][i]$ and $\cell[4][i]$ are filled, and so on, util a $\VD$ is placed in the last unfilled column. When this happens the two lowest rows are cleared and the process continues. 

So we can represent a reachable configuration of a given $n \times m$ board $B$ with a sequence of $m$ integers $(a_1, \dots, a_m)$, where

$$0 \leq a_i \leq \lceil \frac{n}{2} \rceil, \;\;\;   \forall i = 1,\dots, m$$

and $\exists i$ such that $a_i = 0$ (an empty column), with the following mapping: 

$$
\cell = \begin{cases}
   \text{filled}  & \text{if } i \leq  2a_j  \\
   \text{empty}   & \text{if } i >  2a_j
\end{cases}
$$

Each $a_i$ counts the number of vertical pieces placed in the column $i$. For example, in a $10 \times 6 $  board, the sequence $(1,2,0,4,2,3)$ defines the configuration in 
\ref{dom:vconf}.

\begin{figure}[ht]
  \centering
  \begin{subfigure}[b]{0.15\textwidth}
    \centering
    \includegraphics[width=0.9\textwidth]{pictures/dominoes/vertical_configuration.pdf}
    \caption{}
  \end{subfigure}
  \begin{subfigure}[b]{0.15\textwidth}
    \centering
    \includegraphics[width=0.9\textwidth]{pictures/dominoes/vertical_configuration_filled.pdf}
    \caption{}
  \end{subfigure}
    \caption{The $10 \times 6 $ board configuration represented by the sequence $(1,2,0,4,2,3)$, in yellow the 13 dominoes needed to clear the board.}
  \label{dom:vconf}
\end{figure}

\begin{theorem} 
$\textsc{Tetris-NoRotation}\lbrack \VD \rbrack $ \clearing\ is in \pp.
\label{dom:no-rot-vd}
\end{theorem}
\begin{proof}
    Let $B = (a_1, \dots, a_m) $ be the board representation and $k$ the length of the sequence of vertical dominoes. For every constructible board there is an empty column, so the strategy consists on placing each piece in an arbitrary empty column. 

    All the empty cells under the lowest empty row need to be filled to clean the board. Let $a_{\max}$ be the max in the board representation. Since we fill cells with dominoes, the number of dominoes $k_{\min}$ needed to clean the board is:
    $$ k_{\min} = \sum_{i = 1}^m \left( a_{\max} - a_i \right) $$

    If $k < k_{\min}$ we can't clear the board. If $k =  k_{\min}$ we can clear the board. And when $k > k_{\min}$, we can clean the board if after placing $k_{\min}$ dominoes the number of remaining pieces is a multiple of the board width, $k - k_{\min} \equiv 0 \mod m$. Since all the computations can be done in polyatomic time in respect of the input, the problem is in \pp.
\end{proof}

The Figure~\ref{dom:vconf} shows, in yellow color, how the number of pieces needed to clean the board.


\subsection{Tetris with horizontal dominoes}

(treure survival?)

As before, the problem is $\textsc{Tetris-NoRotation}\lbrack \HD \rbrack$, considering both \clearing\ and \survival. Placing a horizontal domino fills two adjacent cells in one row or clears the row, meaning each domino placed can be tracked until the row is cleared. Thus, for any initial constructible configuration, the filled cells can be uniquely grouped into dominoes, allowing us to refer to dominoes instead of individual filled cells.

Additionally, in any constructible board, each row must contain an even number of filled cells. Consequently, when the board width is odd, no row can be cleared. In such cases, the board can only be cleared if it starts as an empty board with an empty sequence of pieces. From this point onward, we will assume that the board has an even number of columns. 

Let $B$ be a board with $m$ columns. We divide the board into $m/2$ \emph{buckets}, where each bucket consists of a pair of consecutive columns. Then:

\begin{lemma}   
    Not placing a domino inside a bucket makes the row unclearable.
\end{lemma}
\begin{proof}
    Let $r$ be a row containing some dominoes. When a domino is placed in a bucket it divides the row into two parts: the cells on the left side of the domino and the ones on the right. Both parts of even length, and containing an even number of filled cells.

    In the other case the two parts have an odd length but containing an eaven number of filled cells, making them impossible to clean
    since there's no way to add an odd number of cells by placing dominoes.
\end{proof}

For example, in the Figure~\ref{dom:buckets}, the second piece occupies the second and the third bucket, making the row un-clearable. 

\begin{figure}[h]
    \centering
    \includegraphics[width=0.2\textwidth]{./pictures/dominoes/buckets.pdf}
    \caption{A board with one partially filled row.}
    \label{dom:buckets} 
\end{figure}


We now can prove both clearing and survival problems.

\begin{theorem}
    $\textsc{Tetris-NoRotation}\lbrack \HD \rbrack $ \clearing\ is in \pp.
\end{theorem}
\begin{proof}


    The input is an $n \times m$ input board $B$, filled with a construable configuration, and sequence of $k$ dominoes $\HD$. If $m$ is odd then the board can't be cleared if $k > 0$ or the initial board isn't empty. 

    When $m$ is even we first need to check if the board is clearable. If there's only one row, checking that the row has been built by placing each piece inside a bucket determines if the row is clearable. When the board has more than one row the same happens. 

    We first group in pieces the filled cells of each row from the initial board. This can always be done because there's no way to clean \emph{"half"} piece. Then we check if each piece is placed inside a bucket. If some piece isn't placed inside a bucket the board can't be cleared. We can compute this in $\mathcal{O}(n\cdot m)$.

    Now the board can be represented with a sequence $(a_1, \dots, a_{m/2})$ of $m/2$ numbers each representing the number of dominoes placed in each bucket.

    $$
    \cell = \begin{cases}
        \text{filled}  & \text{if } i \leq  a_{2j}  \\
        \text{empty}   & \text{if } i >  a_{2j}
    \end{cases}
    $$

    With some $a_i = 0$. Let $a_{\max} = \max \{a_1, \dots a_{m/2}$ \} be the maximum of the sequence. The minimum number of pieces needed to clear the board is:

    $$ k_{\min} = \sum_{i = 1}^{m/2} (a_{\max} - a_i )$$

    If $k < k_{\min}$ the board can't be cleared. If $k = k_{\min}$ the board can be cleared. When $k > k_{\min}$ the board can be cleared if $ k - k_{\min} \equiv 0 \mod m / 2$, sine the remaining pieces have to leave the board empty by filling rows.
\end{proof}

\begin{figure}[ht]
  \centering
  \begin{subfigure}[b]{0.2\textwidth}
    \centering
    \includegraphics[width=0.9\textwidth]{pictures/dominoes/horitzonatl_configuration_1.pdf}
    \caption{}
  \end{subfigure}
  \begin{subfigure}[b]{0.2\textwidth}
    \centering
    \includegraphics[width=0.9\textwidth]{pictures/dominoes/horitzonatl_configuration_2.pdf}
    \caption{}
  \end{subfigure}
  \begin{subfigure}[b]{0.2\textwidth}
    \centering
    \includegraphics[width=0.9\textwidth]{pictures/dominoes/horitzonatl_configuration_3.pdf}
    \caption{}
  \end{subfigure}
  \caption{Some board configurations. In (a) the board can't be cleared because the topmost domino is placed between the first and the second bucket. In (b) the board is represented by the sequence $(2,4,1,0,3)$, it can be cleared. The minimum number of pieces to clean the board is 10, this pieces appear in yellow in (c).}
  \label{dom:horitzonatl_configuration}
\end{figure}

Figure~\ref{dom:horitzonatl_configuration} shows some examples of the above prof. Next follows the \survival. In this scenario the goal is to find a strategy to survive indefinitely, to survive indefinitely.  

\begin{theorem}
    $ \textsc{Tetris-NoRotation}\lbrack \HD \rbrack $ \survival\ is in \pp.
\end{theorem}
\begin{proof}
    
If a given board \( B \) contains a clearable row, we can survive indefinitely by first clearing this row and then continuing to place pieces to refill the topmost row. If no such row exists---for instance, on a board with odd width---there exists a maximum number \( k_{\max} \) of dominoes that can be placed before a loss becomes inevitable. Thus, if the input length \( k \) is less than or equal to \( k_{\max} \), survival is possible; otherwise, it is not. The following procedure checks if any row can be cleared and, if not, computes \( k_{\max} \).

Horizontal dominoes fit naturally in a row, so maximizing the number of dominoes placed on the board \( B \) requires maximizing their placement in each row. To avoid blocking further placements, the board is filled from the bottom to the top. Within a row, a \emph{hole} is defined as a set of contiguous empty cells bounded by filled cells or the board's edges. 

% More formally, for row \( r \), a \emph{hole} is a segment:
%
% \[h^r_{j_1,j_2} = \{\cell[r][j_1], \cell[r][j_1+1], \dots, \cell[r][j_2]\}\]
%
% where all cells in \( h \) are empty, and the boundary cells \( \cell[r][j_1-1] \) and \( \cell[r][j_2+1] \) are either filled or lie outside the board.

To maximize the number of dominoes in the board, each row's holes must be filled as completely as possible. This reduces the problem of filling the entire board to the simpler task of optimally filling individual row holes. By systematically filling holes from the bottom-most row upwards, we ensure the greatest number of dominoes are placed without causing unresolvable blocking and ensure that each hole is above a maximally filled row. 

Let \( h = \{\cell[r][j_1], \cell[r][j_1+1], \dots, \cell[r][j_2]\} \) represent a hole in row \( r \) spanning columns \( j_1 \) to \( j_2 \). All cells in \( h \) are empty, and the boundary cells \( \cell[r][j_1-1] \) and \( \cell[r][j_2+1] \) are either filled or lie outside the board. To place a domino inside the hole, there must exist a trajectory from the top of the board to \( h \), making only reachable holes relevant for consideration.

Using Lemma~\ref{lemma:floating} a hole must take one of the following shapes:

\begin{figure}[ht]
  \centering
  \begin{subfigure}[b]{0.24\textwidth}
    \centering
    \includegraphics[width=0.9\textwidth]{pictures/dominoes/row-holes/simple-hole-1.pdf}
    \caption{}
    \label{dom:hole-a}
  \end{subfigure}
  \begin{subfigure}[b]{0.24\textwidth}
    \centering
    \includegraphics[width=0.9\textwidth]{pictures/dominoes/row-holes/simple-hole-2.pdf}
    \caption{}
    \label{dom:hole-b}
  \end{subfigure}
  \begin{subfigure}[b]{0.24\textwidth}
    \centering
    \includegraphics[width=0.9\textwidth]{pictures/dominoes/row-holes/simple-hole-3.pdf}
    \caption{}
    \label{dom:hole-c}
  \end{subfigure}
  \begin{subfigure}[b]{0.24\textwidth}
    \centering
    \includegraphics[width=0.9\textwidth]{pictures/dominoes/row-holes/simple-hole-4.pdf}
    \caption{}
    \label{dom:hole-d}
  \end{subfigure}
  \label{dom:holes}
\end{figure}

Let \( l = j_2 - j_1 + 1 \) denote the length of the hole \( h \). The maximum number of dominoes that can be placed in holes of type \ref{dom:hole-a} and \ref{dom:hole-c} is \( \lfloor l / 2 \rfloor \) for $l > 2 $, completely filling the hole if \( l \) is even. For a hole of type \ref{dom:hole-b}, the same rule applies for \( l > 4 \). For $l = 4$ only one domino can be placed, as shown in Figure~\ref{dom:hole-4}.

\begin{figure}[h]
    \centering
    \includegraphics[width=0.18\textwidth]{./pictures/dominoes/hole-4.pdf}
    \caption{A type-\ref{dom:hole-b} hole of length 4.}
    \label{dom:hole-4} 
\end{figure}

Finally, the algorithm to decide the problem would look like:
\begin{algorithmic}[1]
    \Function{Tetris-NoRotation[$\HD$]}{$B,k$} \Comment{a board $B$ and $k$ dominos}
    \State $k_{\max}\gets 0$ \Comment{tracks if each hole of a row is fully filled}

    \For{row $r = 1, \dots, n$}
    \State $\text{full} \gets \text{True}$
      \For{hole $h$ in $r$}
        \State $k = \text{MaxDominoes}(h)$ \Comment{the $k$ dominoes that can be maximally placed in $h$}
        \If{$2 \cdot k_r \neq \text{length}(h)$}
          \State $\text{full} \gets \text{False}$
        \EndIf
        \State $k_{\max} = k_{\max} + k$
      \EndFor

      \If{ $\text{full}$}
        \State \textbf{return} $\text{True}$ \Comment{surviving indefinitely}
    \EndIf
\EndFor

\State \textbf{return} $k \leq k_{\max}$
\EndFunction
\end{algorithmic}

The cost of the algorithm is $\mathcal{O}(n \cdot m)$, since looping throught the holes has cost $\mathcal{O}(m)$ and filling the hole maximally has constant cost when the hole shape is known.

\subsection{Clearing the board}

Mes resultats


\end{proof}


