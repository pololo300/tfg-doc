\section{Game Definitions}

Formalization of a Tetris game and definition of the problem. A \emph{Tetris game} will refer to the hole game with its rules, conditions,... and a \emph{tetris match} will refer to a concrete Tetris game instance (no se com dir-ho). We will follow the formalization from \cite{TIH}.


\begin{definition}
  A \emph{board} B is an $n$ by $m$ grid. Each cell $\cell$, $i = 1\dots n$, $j = 1\dots m$ is \emph{filled} or \emph{unfilled}.
\end{definition}

Two conditions are imposed over the initial board, it can't have filled rows and no empty rows can appear under filled cells.  The board will be indexed from bottom to top and from left to right. The cell $\cell[1][1]$ is in the bottom left of the board and $\cell[n][m]$ is the top right-most cell.\\

The following definitions will define the pieces and how they interact with the board.

\begin{definition}
  A \emph{piece type} $t$ is one of the following:$\ALL$. 
\end{definition}

\begin{definition}
 A \emph{piece state} in a board $B$ is a tuple $ P = \piece$ where:
  \begin{itemize}
    \item $t$ is a piece type
    \item $\theta$ is the \emph{orientation}, the number of degrees clockwise from the original piece. $ \theta \in \lbrace 0^\circ, 90^\circ, 180^\circ, 270^\circ \rbrace$. The Figure~\ref{allpieces} shows all the pieces and its rotations.
    \item $\cell$ is the \emph{position} of the piece in of $B$
    \item  $f$ indicates if the piece is \emph{fixed} or \emph{unfixed} in B.
  \end{itemize}

  We will refer to a piece sate as \emph{piece}.
   
\end{definition}


\begin{figure}[ht]
    \centering
    \includegraphics[width=160pt]{pieces/allpieces.pdf}
    \caption{All pieces, in order from top to bottom: $\II$, $\JJ$, $\LL$, $\OO$, $\SS$, $\TT$, $\ZZ$. The first column is the default orientation of a piece upon spawning in; each column to the right indicates a $90^\circ$ rotation clockwise about the rotation center of the piece.}
    \label{allpieces}
\end{figure}

\begin{definition}
  Given a piece type $t$ and a board $B$, the piece $P_0 = \piece[t][0^\circ][n][\lfloor m / 2 \rfloor][\text{unfixed}]$ is the \emph{initial state} of the piece type of $t$ in the board $B$.
\end{definition}

The idea is to have only one active piece in a Tetris match. Only the piece that is moving has a state because when a piece reaches the \emph{fixed} state the piece is automatically merged with the board, and then the next piece starts in its initial state. In order to define how a piece moves thought the board, we need to define some \emph{moves} for changing the piece state.

\begin{definition}
  A \emph{move} is a computable function $m(B, P) = P'$ that given a board $B$ and a piece $P$ outputs a new piece $P'$. The following moves can be applied to an unfixed piece $P = \piece[t][\theta][i][j][\text{unfixed}]$. 
  \begin{itemize}
    \item $r_+$ a \emph{clockwise rotation:} following Figure~\ref{allpieces}, if the rotated piece does not overlap with an occupied cell of the board, the output is 
      $$r_+ (B, \piece[t][\theta][i][j][\text{unfixed}]) = \piece[t][\theta + 90^\circ][i][j][\text{unfixed}]$$

    \item $r_-$ a \emph{counterclockwise rotation:} the same but counterclockwise rotation. 

    \item $s_l$ a \emph{slide to the left:} if all the board cells adjacent to de left of the piece are not occupied, the output is:
      $$s_l (B, \piece[t][\theta][i][j][\text{unfixed}]) = \piece[t][\theta][i][j-1][\text{unfixed}]$$
    \item $s_r$ a \emph{slide to the right:} analogous to the slide to the left.

    \item $d$ a \emph{drop by one row:} if all the board cells bellow the piece are not occupied the output is:
      $$d (B, \piece[t][\theta][i][j][\text{unfixed}]) = \piece[t][\theta][i-1][j][\text{unfixed}]$$

    \item $f$ a \emph{fix:} if any of the board cells bellow the piece is occupied, the output is:
      $$f (B, \piece[t][\theta][i][j][\text{unfixed}]) = \piece[t][\theta][i][j][\text{fixed}]$$
  \end{itemize}
  If the pre-conditions of a move are satisfied, the move is said to be \emph{legal}. If any of the conditions are not satisfied the move is \emph{illegal}.
\end{definition}

All the conditions can be computed in $\mathcal{O}(1)$, because only a constant numbers of cells are need to be visited every time. Let's now define the \emph{trajectory} of a piece type in a Tetris game. Intuitively the trajectory starts when the piece is given to the player, consists on a sequence of legal moves and ends when he fixes the piece in the board. 


\begin{definition}
 Let $B$ be a board and $t$ a piece type. Let $P_0$ be the initial state of the piece type $t$. Then a sequence of $k$ moves $\sigma = (m_1, ..., m_k)$ to the piece state after $i$ moves is a \emph{trajectory} if:

 \begin{itemize}
  \item the move $m_i$ over $P_i$ is a legal move for all $i = 1 \dots k$
  \item and $m_k = f$ is a \emph{fix} move.
 \end{itemize}
 
 Where $P_{i+1} = m_i(P_i)$ is the piece stat in $B$ after $i$ moves.
\end{definition}

The number of moves of a trajectory could be limited by $\mathcal{O}(n \cdot m)$?

With a board and a trajectory we need to define how we merge both. The resulting board will be the merged board. 

\begin{definition}
  Given a board $B$ and trajectory $\sigma = (m_1, ..., m_k)$ of a given piece type $t$ the \emph{merged game board} $B'$ is defined as follows:
  \begin{enumerate}
    \item $B'$ is initially $B$.
    \item The cells of $B$ corresponding to the last piece state of the trajectory are filled in $B'$.
    \item For every filled row $r$ of $B'$:
      \begin{enumerate}
        \item Replace each row $r' > r$ by $r'+1$.
        \item Clear (set all cells to unfilled) of row $m$.
      \end{enumerate}
  \end{enumerate}
\end{definition}

We have now all the components of a Tetris match.

\begin{definition}
  Given a board $B$ and a sequence of $k$ pieces types $P = (t_1,\dots,t_k)$ a \emph{Tetris match} $\Sigma$ is a sequence
  $$ B = B_0, \sigma_1, B_1, \sigma_2, B_2, \dots  \sigma_q, B_q, \; \; q \leq k$$ 
  where:
  \begin{itemize}
    \item $\sigma_i$ is a trajectory of the piece type $t_i$ in the board $B_{i-1}$.
    \item $B_{i+1}$ is the merged board from $B_i$ and $\sigma_i$.
    \item $q < k$ iff doesn't exist any trajectory $\sigma_{q+1}$ from de board $B_q$ with the piece type $t_{q+1}$. In this case we say the game is a \emph{loss}.
  \end{itemize}
\end{definition}

\section{Problem}

All the problems share the same formulation of the problem. The \textsc{Tetirs} problem is:

\begin{itemize}
  \item \textbf{Input} (\textit{a Tetris game}) : $\mathcal{G} = (B,(t_1,\dots,t_k))$ an initial board and a sequence of $k$ pieces types.
  
  \item \textbf{Output} : Does exist a match $\Sigma$ such that $\Phi ( \mathcal{G}, \Sigma )$ holds? 
\end{itemize}

Where $\Phi(\mathcal{G},\Sigma)$ is computable objective function that only takes into account the final state of the pieces, ignoring the piece trajectory (\emph{checkable and acyclic}). 


We will divide the problem variants classification in two parts: variations on the objective function and variations on the game formulation.

\subsection{Objective functions} 

Will refer to the \textsc{Tetris} problem with the objective function $\Phi$ as $\textsc{Tetris} [ \Phi ]$. The following are objective functions:

\begin{itemize}
  \item \texttheo{k-cleared-rows}: in the game $\mathcal{G}$ , does $\sigma$ clear at least $k$ rows?
  \item \texttheo{k-tetrises}: in the game $\mathcal{G}$ , does $\sigma$ contain at least $k$ tetrises?
  \item \texttheo{h-height-filled}: in the game $\mathcal{G}$ , does $\sigma$ never fill a gridsquare above height $h$?
  \item \texttheo{p-placed-pices}: in the game $\mathcal{G}$ , does $\sigma$ place at least $p$ pieces before losing.
\end{itemize} are all in \npc (\cite{TIH}). The corresponding optimization problems are also \npc.

The two most popular variations are:

\begin{itemize} 
  \item \label{var:clearing} \texttheo{clearing}: in the game $\mathcal{G}$, does $\sigma$ leave every cell of the board unfilled?
  \item \label{var:survival}\texttheo{survival}: is equivalent to \texttheo{n-height-filled}, where $n$ is the height of board $B$. This is finding a sequence $\Sigma$ that is not a loss.
\end{itemize}

\subsection{Game variations}

A game variation is changing some parameter that appears during the definition of the game. When showing results, if the objective function is not specified, the result is assumed to be valid under \texttheo{survival} and \texttheo{clearing}. The following parameters of the game can be changed: 

\begin{itemize}
  \item \textbf{Pieces:} changing the pieces set of a Tetris game
  \item \textbf{Moves:} extending, reducing or changing the moves that can be applied to the pieces. 
  \item \textbf{Board:} fixing the board dimensions or impose restriction of the initial board.
  \item \textbf{Pieces sequences:} imposing restriction over the sequence of pieces.
\end{itemize}

\subsubsection{Pieces variations:}

All variations in this section consists on changing the \emph{piece types}. The problem formalization is the same but, if new pieces apear, appropriately changing the \emph{move} functions for the new pieces types. 

\begin{itemize} 
  \item  \textbf{1 piece:} for any 1-size subset of $\{\ALL\}$ the problem is open.
  \item  \textbf{2 pieces:} for any 2-size subset of $\{\ALL\}$ \texttheo{clearing} and \texttheo{survival} is \npc. (under SRS) \cite{TWFP}.
  \item  \textbf{monomies:} \cite{TT}
  \item  \textbf{domonies:} \cite{TT}
  \item  \textbf{trimones:} \cite{TT}
  \item  \textbf{k-monies:} \cite{TT}
  \item  \textbf{rectangular pieces:} \cite{TWRP}

\end{itemize}

\subsubsection{Moves variations}

(Explica el SRS)

Changing the set of available moves that a player can apply to a given piece we obtain different variants.

\begin{itemize}
  \item \textbf{No rotation:} removing the rotation moves.

  \item \textbf{Hard Drop:} when playing the Tetris video game, specially in the initial rounds of a game, the player can hard drop a piece in order to avoid dropping the piece one row at a time. \emph{drop} and \emph{fix} are removed from the set of moves, and \emph{hard-drop} is added. When applied this move, the piece moves maximally downward before fixing into place.

    \begin{itemize}
      \item Tetris with Hard Drop is \npc  even if the pieces set is restricted to $\{ \II, \OO \}$ \cite{TWFP}. 
      \item Tetris with Hard Drop is \nph  even if the pieces set is restricted to $\{ \II, \OO, \TT\}$ \cite{TWFP}. 
    \end{itemize}
    

  \item \textbf{20-G:} the pieces in a Tetris video game fall faster as the game progresses. (explicacio). 

    \begin{itemize}
      \item Tetris with 20-G is \npc  even if the pieces set is restricted to $\{ \II, \OO \}$ \cite{TWFP}. 
      \item Tetris with 20-G is \nph  even if the pieces set is restricted to $\{ \II, \OO, \TT\}$ \cite{TWFP}. 
    \end{itemize}

  \item \textbf{Hold-Function:} in this version the player has a \emph{box} to put aside a piece for a later use. Open problem.
\end{itemize}



\subsubsection{Board variations}

In the main problem the board $B$ is an $n$ by $m$ grid, fixing rows or columns to a constant value gives different problems. \cite{TCB}.


An open problem posted in some papers is to determine the complexity when the board is empty.

Explica millor.

\subsubsection{Pieces sequences:}

7-bag randomizer. random generation of the sequence. open problem. Explica millor.


\section{Related problems}

Other problems related to Tetris:

\begin{itemize}
  \item Tetris Decidability: \cite{TAD}
  \item Tetris Competitive:  \cite{TINC}
  \item Tetris strategies:  \cite{TINC}
\end{itemize}
