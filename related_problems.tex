\section{Related problems}

Other problems related to Tetris.

\subsection{Tetris and decidability}

In \cite{TAD}, they consider a variant of Tetris where the sequence of pieces (together with their orientation and horizontal position, which cannot be changed anymore) is generated by a finite state automaton. They show that it is undecidable, given such an automaton, and starting from an empty game board, whether one of the generated sequences leaves an empty game board.

Since we are dealing with pieces sequences of a regular language we cannot fix the board height, so in this version the board height is unbounded. The formal problem definition is:

\vspace{1em}
\textbf{Instance:} an empty board $B$ of width $w$ and $L$, a regular language describing sequences of Tetris pieces with their initial position and orientation. 

\textbf{Question:} is there a sequence in $L$ that leaves the game board empty after dropping all pieces into the empty board?
\vspace{1em}

This problem with a board of width of 10 is undecidable for sequences only of $\II$ (reduction from PCP), but with only $\OO$ pieces the problem is decidable.


Introducing user intervention, letting the piece move while falling, we have the that the problem is decidable is the pieces types are restricted to $\{ \OO , \II \}$, or if the pieces are restricted to one piece type for arbitrary board width.

\subsection{Tetris is not Competitive}

Explores Tetris as an online game, where the player can only see the $l$-upcoming pieces (\emph{lock ahead}), in contrast with an offline game where the player knows completely the piece sequence.

In online-offline game studies, the goal is to compare an optimum offline strategy with an offline one. In this paper they construct pieces sequences without any loss-avoiding strategies for players with arbitrary (finite) lock ahead, but a loss-avoiding strategy exists in the offline version.

Next they use the before construction to show that an online player performs arbitrarily badly against an offline player for some optimization goals.

\subsection{How to construct Tetris configurations}

The main result of \cite{HTCT} is that every reasonable Tetris configuration is constructible, if a simple parity condition on the configuration is met.

To this problem we can attach a decision problem: given a configuration and an ordered series of Tetris pieces, their sizes satisfying a suitable congruence modulo 4; is it possible to construct the configuration using this series? 

We can also try to minimize the number of pieces needed to build a Tetris configuration.

\subsection{How Fast Can We Play Tetris Greedily With Rectangular Pieces?}

\cite{TWRP} considers a Tetris variant with a board of width $w$ a and an infinite height, where the pieces are rectangles of arbitrary integer dimensions and rotations are not allowed. The model definition is:

\begin{definition}[RDDS]
  A \emph{Rectangle Dropping Data Structure} maintains a set of $O(n)$ independent axis-aligned rectangles in the plane with integer coordinates, lying on or above the $x$-axis and between the vertical lines $x = 0$ and $x = w$, and allows the following:

  \begin{itemize}
    \item \emph{Preprocessing:} Initialize an empty RDDS containing no rectangle.
    \item \emph{Update:} Given an axis-aligned rectangle $R$ and a non-negative integer $x_d$, drop $R$ with left border at $x$-coordinate $x_d$ (here we assume that $R$ and $x_d$ are such that $R$ will lie between the lines $x = 0$ and $x = w$).
    \item \emph{Query:} Given an axis-aligned rectangle $R$, return the position where $R$ must be dropped to end up as low as possible (or one such position if it is not unique) as well as the height of the highest point in the set of rectangles which would result from that move.
  \end{itemize}
\end{definition}

RDDS is assumed to be a \emph{world-RAM machine}, an abstract machine similar to a random-access machine, but with finite memory and word-length. It works with words of size up to $ w $ bits, meaning it can store integers up to $2^{w} - 1$. 
The model is useful for analyzing the complexity of algorithms in a more realistic context, where the execution time not only depends on the amount of data but also on the representation of that data and the operations performed on it. 

They show that on a board with width $w = O(n)$, both operations of RDDS  cannot be supported in time $O(n^{1 / 2 - \epsilon})$ simultaneously if OMv (online matrix-vector multiplication problem) conjecture is true. 


\subsection{Why Most Decisions Are Easy in Tetris?}

Aquest mel vull llegir. Diu que el tetris no es especial.
